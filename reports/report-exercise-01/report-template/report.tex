\documentclass[10pt,a4paper]{article}

%%%%%%%%%%%%%%%%%%%%%%%%%%%
% MODIFY:
\usepackage{graphicx} 
\usepackage{subfigure}
\usepackage{float}
\newcommand{\authorA}{Alejandro Hernandez Artiles (03785345)}
\newcommand{\authorB}{Pavel Sindelar (03785154)}
\newcommand{\authorC}{Haoxiang Yang (03767758)}
\newcommand{\authorD}{Jianfeng Yue (03765255)}
\newcommand{\authorE}{Leonhard Chen (03711258)}
\newcommand{\teamlead}{~\textbf{(Project Lead)}}
\newcommand{\groupNumber}{C} % - YOUR GROUP NUMBER
\newcommand{\exerciseNumber}{1} % - THE NUMBER OF THE EXERCISE
\newcommand{\sourceCodeLink}{https://github.com/alejandrohdez00/Exercises-MLCMS-Group-C/tree/main/Exercise-1}

\newcommand{\workPerAuthor}{
\authorA\teamlead & Task 1&20\%\\
      &Task 2&20\%\\
      &Task 3&20\%\\
      &Task 4&20\%\\
      &Task 5&20\%\\
      \hline
\authorB & Task 1&20\%\\
      &Task 2&20\%\\
      &Task 3&20\%\\
      &Task 4&20\%\\
      &Task 5&20\%\\
      \hline
\authorC & Task 1&20\%\\
      &Task 2&20\%\\
      &Task 3&20\%\\
      &Task 4&20\%\\
      &Task 5&20\%\\
      \hline
\authorD & Task 1&20\%\\
      &Task 2&20\%\\
      &Task 3&20\%\\
      &Task 4&20\%\\
      &Task 5&20\%\\
      \hline
\authorE & Task 1&20\%\\
      &Task 2&20\%\\
      &Task 3&20\%\\
      &Task 4&20\%\\
      &Task 5&20\%
}

%%%%%%%%%%%%%%%%%%%%%%%%%%%

\input{./imports.tex}

\begin{document}

\frontpage

\paragraph{Introduction}
 In this project we introduce ourselves to the field of human crowd modelling using a simple cellular automaton, a useful approach for modelling complex and dynamical systems. The automaton works by having cells follow certain rules to determine their next state. Despite the simplicity of these rules, their iteration in different cells produces a complex and interesting behaviour.

\paragraph{}Each cell of our cellular automaton can be in four different states: "Empty", "Pedestrian", "Obstacle" and "Target". In the system, the pedestrians will try to reach a target cell, avoiding the obstacles by means of update scheme, which basically tries to reduce the Euclidean distance of each pedestrian to one of the target cells by updating the position of the pedestrian to the nearest neighbouring cell to a target. This update scheme could be changed during the project.

\begin{task}{1, Setting up the modeling environment}
In this first task we want to setup the basic visualization for running the simulations. The initial project provides a simple GUI for the simulation of the cellular automaton. Each cell draws pedestrians, targets, obstacles as a colored rectangle and by clicking the step button a single time-step is simulated. We use the following colors for each object:
\begin{itemize}
    \item White for empty cells
    \item Blue for target cells
    \item Magenta for obstacle cells
    \item Red for cells with pedestrians
    \item Light Grey for cells traversed by a pedestrian (New)
\end{itemize}
While the example project does provide a rough framework, we added several major changes to allow flexibility for adding new functionality and ease of use for the following tasks. We will first explain the implementation decisions and then at the end of this report section we will provide instructions on how to use the program. To improve the initial setup the main goals were:
\begin{enumerate}
    \item to define a file format for simulation scenarios,
    \item to implement loading scenarios from external files and
    \item to implement more controls for performing simulations.
\end{enumerate}

\paragraph{1. Define a file format for simulation scenarios} 
After observing how a simple scenario was hard-coded inside the \texttt{start\_gui(...)} function in \texttt{gui.py}, we first defined the format of the scenario files. All simulation scenarios are stored in \texttt{.json}-files and their format is as follows:

\begin{verbatim}
{
    "iterations": n,
    "cell_size": [width, height],
    "targets": [
        [tar_x, tar_y]
    ],
    "obstacles": [
        [obs_x, obs_y]
    ],
    "pedestrians": [
        { "position": [pos_x, pos_y], "speed": v }
    ]
}
\end{verbatim}

\newpage
\begin{itemize}
    \item \texttt{"iterations"} Defines the integer $n$ as the number of simulation time-steps that are performed automatically after starting the simulation
    \item \texttt{"cell\_size"} Defines the integers \texttt{width} and \texttt{height} for the cells of the cellular automaton. Note that GUI will display all ratios between width and height in a square canvas, hence an uneven ratio will lead to a stretched display of cells.
    \item \texttt{"targets"} Defines a list of target coordinates stored as an integer list.
    \item \texttt{"obstacles"} Defines a list of obstacle coordinates stored as an integer list.
    \item \texttt{"pedestrians"} Defines a list of pedestrians, each containing integer starting positions and a float velocity. Note that multiple pedestrians can be placed in the same cell.
\end{itemize}

\paragraph{2. Implement loading scenarios from external files}
To load these scenario files, we implemented both loading from CLI and GUI. Loading from CLI requires us to modify \texttt{main.py}, so we can parse one additional optional arguments when starting the program. The arguments allow us to load a specific scenario file, choose a specific algorithm and whether we want to pedestrians to avoid each other. By default the scenario \texttt{scenario\_task1.json} is loaded with the fast marching algorithm and pedestrian avoidance enabled. All of these arguments are passed to the GUI class.

Inside the GUI object the simulation of the scenario is setup in 2 steps:
\begin{enumerate}
    \item The \texttt{start\_gui(...)} function initially loads the contents of the scenario file inside the class attribute \texttt{config=None}, which is later on needed for resetting and loading a scenario. (Similarly many other variables created in \texttt{start\_gui(...)} are added as class attributes to shorten the signature of functions inside this class.)
    \item Then \texttt{self.load\_scenario\_from\_config(...)} is called to actually setup the simulation environment with the given parameters from the scenario file.
\end{enumerate}
To load from the GUI we perform these same 2 steps inside the \texttt{load\_scenario(...)} function.

\paragraph{3. Implement more controls for performing simulations}
For debugging purposes the simulation was changed to run automatically after pressing the start button. Additionally we added a reset button to reload the starting configuration of the scenario file. On the top right of the program the remaining iterations are displayed. Once the simulation has ended the program will display "Simulation finished". Additionally on the bottom left the change background button visualizes the the distance values with a blue hue.

\paragraph{Instructions} To define a scenario file follow the file format defined in the previous section. The program can be started from CLI and optionally a scenario file, algorithm choice and pedestrian avoidance flag can be passed to initially load it into the simulation. Possible algorithm arguments are "F" for fast marching (default), "D" for Dijkstra and "S" for the default update rule. To disable pedestrian avoidance adding the flag \textit{--ignorePedestrians} suffices. The default starting command for task 1 is as follows:

\begin{verbatim}
python main.py --scenario scenario_task1.json --algorithm S --ignorePedestrians
\end{verbatim}

After the simulation has loaded there are 5 buttons available.
\begin{itemize}
    \item \texttt{Load simulation}: Opens a dialogue window. Selecting a .json file loads a scenario into the program. 
    \item \texttt{Start simulation}: Starts the simulation and stops once the pedestrians arrive at their target or the iteration counter on the top right reaches 0.
    \item \texttt{Step simulation}: Simulates a single time-step of the pedestrians and pauses the current running simulation.
    \item \texttt{Reset simulation}: Resets the simulation to the initial state of the scenario.
    \item \texttt{Change background}: Changes the background to display the distance to targets. It takes into account the distances computed by the algorithm in use.
\end{itemize}

The following figures provide an example of using the simulation controls with \texttt{scenario\_task1.json}, which is the scenario provided in the example project. Adding obstacles is possible, but will be demonstrated in task 2. Figure \ref{task1}a shows the dialogue window when loading a new simulation. Figure \ref{task1}b shows an example scenario with 3 pedestrians, 3 targets and objects. In figure \ref{task1}c we see the result while running the simulation. In figure \ref{task1}d the pedestrians have reached their targets by either manually performing simulation steps or running the simulation automatically. In Figure \ref{task1}e the visualization of distances to the targets has been enabled.

\begin{figure}[H] 
\centering
\subfigure[Load simulation]{
\includegraphics[width=0.4\textwidth]{report-template/image/Task1_1.png}}
\subfigure[Start simulation]{
\includegraphics[width=0.4\textwidth]{report-template/image/Task1_2.png}}
\subfigure[Simulation Finished]{
\includegraphics[width=0.4\textwidth]{report-template/image/Task1_3.png}}
\subfigure[Target reached]{
\includegraphics[width=0.4\textwidth]{report-template/image/Task1_4.png}}
\subfigure[Visualizing target distances]{
\includegraphics[width=0.4\textwidth]{report-template/image/Task1_5.png}}
\caption{Use of the GUI}
\label{task1}
\end{figure}
\end{task}

\newpage
\begin{task}{2, First step of a single pedestrian}
In the  second task we want to test the basic functionalities of the simulations in a simple scenario. Here we created a new scenario file that satisfies the following  specifications:
\begin{itemize}
    \item Iterations: 25
    \item Cell size: 50x50
    \item Target position: (25, 25)
    \item Pedestrian position: (5, 25)
\end{itemize}
The scenario can be opened via the load simulation button and selecting "scenario\_task2.json" in the dialogue window. To run this test from CLI run the following command:
\begin{verbatim}
python main.py --scenario scenario_task2.json --algorithm S
\end{verbatim}
Afterwards, the simulation finishes with the pedestrian reaching the target and waiting there.
\begin{figure}[H] 
\centering
\subfigure[Initial state]{
\includegraphics[width=0.4\textwidth]{report-template/image/Task2_1.png}}
\subfigure[Final state]{
\includegraphics[width=0.4\textwidth]{report-template/image/Task2_2.png}}
\caption{Simulation of task 2}
\end{figure}

In the Figure \ref{pedestrian_block} we make certain cells inaccessible by adding a vertical line of obstacles to block the pedestrian. Here we still use the standard movement algorithm with Euclidean distance. We treat all 8 connected neighboring cells as the "next step" to calculate the optimal step. As the pedestrian reached the cell neighbouring to the obstacles, the pedestrian get stuck at that point. Use the following command to reproduce this example.

\begin{verbatim}
python main.py --scenario scenario_task2a.json --algorithm S
\end{verbatim}

\begin{figure}[H] 
\centering
\includegraphics[width=0.4\textwidth]{report-template/image/Task1_7.png}
\caption{Pedestrian blocked by obstacle}
\label{pedestrian_block}
\end{figure}
\end{task}

\begin{task}{3, Interaction of pedestrians}
The third task serves to manage the speeds of pedestrians. To do this, we created a scenario where 5 pedestrians are at approximately the same distance from a target, forming a circle. In particular, as we want to check the difference in speed between horizontal, vertical and diagonal movements. In the scenario there are four pedestrians forming a cross with respect to the centre, providing the horizontal and vertical movements, and we have placed a pedestrian at the same distance on a diagonal.
\begin{figure}[H] 
\centering
\subfigure[Initial state]{
\includegraphics[width=0.45\textwidth]{report-template/image/Task_3_1.png}}
\subfigure[Final state]{
\includegraphics[width=0.45\textwidth]{report-template/image/Task_3_2.png}}
\caption{Initial simulation of task 3}
\label{init_sim_3}
\end{figure}

As we can observe in Figure \ref{init_sim_3}b, the diagonal path to the target is faster than the horizontal and vertical ones. That is because our system is using a discrete grid, but the distance from a pedestrian to the target is computed using the Euclidean distance. Therefore, if a pedestrian moves horizontally or vertically, it is advancing a distance of 1, but if it moves diagonally it is advancing a distance of $\sqrt{2} = 1.41...$, thus travelling more distance in the same time. 

We tried to solve the problem by putting a penalty on the diagonal movement. We calculated the Euclidean distance travelled by a pedestrian and when a series of time steps were fulfilled we calculated its velocity as the radius of the distance and the number of those time steps. If the pedestrian's speed was greater than the desired speed, entered as a parameter in the scenario's .json file, it was forced to stop at the next time step. As we can observe in Figure \ref{sim_task3}a, this method achieved that all pedestrians reached the target at the same time step, reducing the difference in their speeds. However, it is not very realistic for a crowd to be constrained in diagonal motion only, so we generalised this concept and made the motion of pedestrians stochastic, still trying to penalise more the diagonal movements. We achieve this using formula (1):
 \begin{equation}
\frac{\text{desired\_speed}}{\text{dist}(\text{position}, \text{min\_cost\_pos})} < \epsilon
\end{equation}
where \(\epsilon\) is a random number \(\epsilon \in [0,1]\). This formula is checked inside the \texttt{update\_step} function in the Pedestrian class, and if the condition is fulfilled, the pedestrian does not advance a position in that time step. Basically, the pedestrian's speed becomes a probability distribution, since we are comparing the ratio between their desired speed and the Euclidean distance to a random number. 

If the Euclidean distance from the pedestrian's current position and its next position is greater (it is diagonal) it is more likely to stop than making a horizontal or vertical movement. On the other hand, as the divisor is the desired velocity, if this is very large, it will be less likely to stop as well. Therefore, in this simple way a velocity distribution can be modelled for the pedestrians with the diagonal motion constraints.

In Figure \ref{sim_task3}b we can observe a non-uniform speed distribution allowing horizontal and vertical paths to be equally fast as diagonal ones. In Figure \ref{sim_task3}c is shown the comparison between the desired speed and the actual speed of each of the pedestrians. The stochastic distribution of speeds can be noticed. For example the 3 fastest pedestrians in Figure \ref{sim_task3}b have a higher current speed and are at the top of the graph, and the last two pedestrians have a current speed of approximately 0.1 less.
\begin{figure}[H] 
\centering
\subfigure[First implementation]{
\includegraphics[width=0.4\textwidth]{report-template/image/Task_3_3.png}}
\subfigure[Stochastic implementation]{
\includegraphics[width=0.4\textwidth]{report-template/image/Task_3_4.png}}
\subfigure[Actual speed vs desired speed distribution]{
\includegraphics[width=0.4\textwidth]{report-template/image/plot_task3.png}}
\caption{Implementations of task 3}
\label{sim_task3}
\end{figure}

Pedestrian avoidance was not needed for this exercise, however it is still a feature of pedestrian interaction. Pedestrian avoidance is explained in the next section.
Use the following command to run the experiment:
\begin{verbatim}
python main.py --scenario scenario_task3.json --algorithm S 
\end{verbatim}
\end{task}


\begin{task}{4, Obstacle avoidance}
The starting point was to create a heap with decrease key data structure. The decrease function was implemented through integer IDs. After every add an ID was returned (simply the number of times add was called on the particular heap beforehand), this seams like a cleaner implementation than using the python id method or some hashing function as it is language oblivious, doesn't necessitate keeping the original object around and allows for the same values to occur multiple times with different keys if necessary. The first implementation was a simple list heap with O(N) time complexity. This was later changed into a binary tree in list heap.

The Dijkstra's algorithm implementation was originally a path finding one because the author had miss-read the problem definition. The author was of the belief that it was necessary to calculate the distance from each position to the targets by starting the Dijkstra's algorithm at the position while it is of course elementary to instead start at the target and work one's way backward. In hopes of somehow decreasing the substantial runtime this caused the author tried to make the Heap more efficient by using numpy arrays instead of lists of tuples. Turning the algorithm into an A* by adding a euclidian distance heuristic was also tried. In the end it turned out this was not enough and the author soon discovered his error and that these additions were not necessary, so they were removed from the final code. The implementation was made modular by making the distance and neighborhood functions it's parameters of type typing.Callable.

For implementing the Fast Marching Method algorithm all that was necessary was to realize that having implemented Dijkstra's algorithm all that was required to implement the Fast Marching Method was to add a dictionary with distances to closed keys as a parameter to the distance function. All that was needed then was to pass a Fast Marching Method distance update function as a distance function to the Dijkstra's algorithm implementation.

The Fast Marching Method distance update is based on the simple idea that if the derivative of the distance between neighboring nodes is to be constant the identity
$$max(D_{0,0}-D_{1,0}, D_{0,0}-D_{-1,0})^2 + max(D_{0,0}-D_{0,1}, D_{0,0}-D_{0,-1})^2 - C = 0  $$
must hold. Where $D_{0,0}$ is the distance to the updated node at position a,b and $D_{x,y}$ is simply the node at position (a+x,b+y). C being the square of the derivative of the distance function which in our case where we use the Fast Marching Method only for distance calculation is an arbitrary positive number, though 1 seems to work the best as at larger numbers the method's inaccuracy increases.
All we need to get the update is to get the greater of the roots of this polynomial.
In case there is no such root we simply use the approximate expression
$$\frac{-b}{2a}$$
where b is the first power multiplier in the polynomial and a the second power one.

In cases where there is another value in the direction of the maximum distance we can use the second order Fast Marching Method, this has been implemented in the project but describing the theory behind it seems out of scope of this work.

The user interface function \verb+path_lengths+ was for the user's ease of use made to only take a neighborhood function and \verb+is_obstacle+ function, with a simple enum used to choose between different implementations.

As stated before using this implementation in the application was as simple as calling the function for every target with the neighborhood function set to either adjacent for the Fast Marching Method implementation or adjacent+diagonal for the Dijkstra's algorithm implementation, the obstacle function simply checking whether the grid contains and obstacle at given position and recording the minimum resultant distance for each cell.

\paragraph{Pedestrian avoidance}
As instructed we only take into consideration obstacles and ignore pedestrians while calculating distances. This significantly reduces computational complexity but forces any pedestrian avoidance to have a purely greedy character. Of course in most ways this is exactly the behaviour that makes most sense as humans are generaly very poor at predicting the behaviour of other humans in a crowd over any longer period of time. We implement this through a simple cost function. To calculate this for cost for a target position the algorithm takes the count of pedestrians at a given surrounding position and multiplies them with weights we assign based on the offset from the target position. These weights are a dictionary of weights decreasing exponentially with euclidian distance of the key from the center of the matrix.

$$
f(d) =
     \begin{cases}
       exp(\frac{p}{d^2-d_{max}^2}) \ \text{if $d < d_{max}$}\\
       0 \ \text{otherwise}\\
     \end{cases}
$$

$p$ being the parameter determining how fast $f$ decreases.

To allow for easier manipulation these values are then normalized.

Setting $p$ too low can cause clustering into lines or points since the small distance positions occur less frequently than the high distance ones. It would therefore seem reasonable that at least half the distribution be at the position (0,0).

In general because the complexity of the calculation increases with the square of $d_{max}$ with the python implementation values greater than 10 decrease the speed of the simulation. Then again multipliers of less than 0.01 (which such a distribution would have even were it uniform) hardly contribute much. These are especially preelent if we want at least half the distribution be near the center. Therefore if we explicitly ignore such small multipliers most reasonable distributions of any size are computationally reasonable for given pedestrian counts. This needs pre-computing the values and caching them instead of calculating the coefficient during every cost calculation as it was originally, this doesn't have much impact on the complexity neither positive or negative (apart from the aforementioned ability to ignore small coefficients). It also allows for the code to be more modular as we can theoretically pass any coefficient dictionary to the function.

\paragraph{Experiment description}
In this experiment we perform the RiMEA-12 test according to \cite{rimea2016}, where pedestrians have to pass a tight hallway. This experiment should show that pedestrians will form a congestion only at the first bottleneck. Notably here are the different CLI prompts we can use to simulate different pedestrian avoidance behaviour. By default we run the following command:
\begin{verbatim}
python main.py --scenario scenario-RIMEA-12.json
\end{verbatim}
Use the flag \verb+--multiplePedestriansInOneCell+ to change the behaviour of pedestrian avoidance. If the flag \verb+--ignorePedestrians+ is set before pedestrian avoidance will be completely disabled, hence setting the flag \verb+--multiplePedestriansInOneCell+ does nothing. 

If the simulation with no flags is run we observe the following figures \ref{rimea12}a, \ref{rimea12}b and \ref{rimea12}c to show the behaviour in the bottleneck scenario. When presented with the bottleneck the pedestrians are forced to wait as the cost of being forced through the bottleneck all at once is too high. The pedestrians also immediately disperse again after leaving the bottleneck to minimize their cost.

\begin{figure}[H] 
\centering
\subfigure[Congested pedestrians]{
\includegraphics[scale=0.4]{image/Bottleneck-start}}
\subfigure[Pedestrians flowing through after bottleneck ended]{
\includegraphics[scale=0.4]{image/Bottleneck-trickle}}
\subfigure[Bottleneck end]{
 \includegraphics[scale=0.4]{image/Bottleneck-end}}
\subfigure[Bottleneck speeds]{
\includegraphics[scale=0.45]{report-template/image/bottleneck-speeds.png}}
\caption{RiMEA-12 simulation}
\label{rimea12}
\end{figure}

It seems that the blockage causes the actual speeds to not correlate as much with the desired speeds. We can observe this better if we don't use pedestrian avoidance in the same scenario. In Figure \ref{rimea12withoutped}a and \ref{rimea12withoutped}b we can see that now there is no bottleneck and pedestrians follows a straightforward path to the target, with no need for waiting the other pedestrians. This saved time is exhibited in Figure \ref{rimea12withoutped}c, where the speeds are more uniformly and linearly distributed, showing that pedestrians have made the way according to their desired speed, instead of reducing the actual speed due to the bottleneck. Also, analyzing the graph, we can observe that, naturally, without bottleneck there is more density of pedestrians with equivalent desired\_speed/actual\_speed ratio, as pedestrians can move freely. The maximum density in Figure \ref{rimea12}d is 3 and in \ref{rimea12withoutped}c is 6.
\begin{figure}[H] 
\centering
\subfigure[No bottleneck]{
\includegraphics[scale=0.4]{report-template/image/rimea12withoutped.png}}
\subfigure[More advanced step in simulation]{
\includegraphics[scale=0.4]{report-template/image/finalwithoutped.png}}
\subfigure[No bottleneck speeds]{
\includegraphics[scale=0.45]{report-template/image/rimea12graphnoped.png}}
\caption{RiMEA-12 simulation without pedestrian avoidance}
\label{rimea12withoutped}
\end{figure}

Then, in the chicken test if we simply use distance to target as path to target cost, thus ignoring obstacles, the pedestrian gets stuck at the first intersection of a line to the target and the obstacle (Figure \ref{chickentest}a). However, if we calculate the distance to the target using Fast Marching, the pedestrian can find an optimal path (Figure \ref{chickentest}c). Going deeper, by changing the background to the computed distances by the algorithm in use, we can analyse the cause of the problem, as it can be seen in Figure \ref{chickentest}b and \ref{chickentest}d. The pedestrian has a more informed distance measurement using the Fast Marching Method. The CLI prompts for \ref{chickentest}a and and \ref{chickentest}c are respectively:
\begin{verbatim}
python main.py --scenario chicken_test.json --algorithm S

python main.py --scenario chicken_test.json --algorithm F 
\end{verbatim}


\begin{figure}[H] 
\centering
\subfigure[Chicken test no path finding]{
\includegraphics[width=0.35\textwidth]{report-template/image/chicken-test-dumb.png}}
\subfigure[Standard algorithm visual test]{
\includegraphics[width=0.35\textwidth]{report-template/image/chicken_dumb_vis.png}}
\subfigure[Chicken test with path finding]{
\includegraphics[width=0.35\textwidth]{report-template/image/chicken-test-clever.png}}
\subfigure[FMM visual test]{
\includegraphics[width=0.35\textwidth]{report-template/image/fmm-visual-test.png}}
\caption{Chicken test simulation}
\label{chickentest}
\end{figure}


\end{task}




\newpage
\begin{task}{5, Discussion of future work and other approaches (5\%)}
The paper \cite{korbmacher2022review} talks in the discussion and future work session about new trends in the approximation of pedestrian dynamics. 

\paragraph{Hybrid approaches} They mention the hybrid approaches that mix neural networks and knowledge-based approaches as a very valuable alternative, as many drawbacks of DL algorithms are strengths in the knowledge-based approach, like the interpretability of its parameters. They describe two ways of combining the two concepts:
\begin{itemize}
    \item \textbf{Use Knowledge-based models to improve DL algorithms} Some interesting examples of these are:
    \begin{itemize}
    \item \textbf{Data Generation:} Knowledge-based models generate synthetic data sets to augment training data for the DL algorithm, particularly useful when real-world data is limited or lacks diversity. For instance, simulations based on KB models can enhance training data for scenarios not well-represented in existing data sets \cite{von2020combining}.
    \item \textbf{Knowledge-Guided Design of Architecture: }Knowledge-based is integrated into the architecture design of DL models to capture domain-specific dependencies among variables. This involves embedding knowledge directly into the network structure \cite{antonucci2020generating}.
    \item \textbf{Knowledge-Guided Loss Function:}  Knowledge is incorporated into the loss function of DL algorithms to ensure outputs adhere to physical laws, minimizing unrealistic predictions \cite{willard2022integrating}. 
    \end{itemize}
    \item \textbf{Use DL algorithms to improve the prediction of knowledge-based models:} This can be done for example by training to correct errors made by KB models by predicting model residuals, as in \cite{willard2022integrating} too. Also, DL algorithms can be used to adjust the parameters of KB models, as they did to refine the social force model parameters in \cite{gottlich2020artificial}. 
\end{itemize}

As we can see the authors conclude that neural network and knowledge-based approaches can be used as complementary methods to strengthen each other's weaknesses.

As we saw in Task 2, we tried an hybrid approach as it seemed like an interesting way to try to improve our results with the neural net. Following the different types of hybrid models suggested by the paper, ours is closer to the former, and within it, closer to the \textit{knowledge-guided design of architecture} as we are changing the last layer to resemble the Weidmann model. However, the hybrid approach we designed was not that better compared to the rest, being a little bit better than other models we tried. This could happen also because our hybrid model is very simple and rudimentary, nevertheless we think that, in general, hybrid models would be refined in the future leading to better results, as they have "the better of both worlds".

On the other hand, in \cite{korbmacher2022review} they also discussed other directions for the future. For example, Reinforcement learning algorithms are being used, as they sometimes resemble how people learn and move. Reinforcement learning is a common method also to made agents learn about its environment by setting a reward for reaching a goal while avoiding collisions. However, RL methods need a reward function, that sometimes is not easy to model, and a priori goal or destination. There are some works where they combine RL and DL approaches to try to mitigate these drawbacks \cite{everett2021collision}.

Finally, the propose some questions regarding the possibility of a successful use of DL algorithms in large-scale crowd simulations. In our opinion, as we have observed in our report and experience with the previous tasks, we think that one of the main problems of DL is that it needs a lot of data. Moreover, we have seen that it is difficult for them to learn more insightful patterns of the pedestrian dynamics. Specifically, when trained on a specific scenario, they tend to learn the dynamics unique to that scenario rather than general patterns of movement. Consequently, when tested in a new scenario, performance tends to degrade. This limitation in generalization in new environments is common to neural networks across various domains, but it holds particular significance in crowd simulation due to the multitude of potential scenarios. We hope that in the future, advancements in deep learning will address this limitation, making it even more beneficial for crowd simulation tasks.
\end{task}

\bibliographystyle{plain}
\bibliography{Literature}

\end{document}